\documentclass[12pt]{article}

\usepackage[CJKnumber]{xeCJK}
\setCJKmainfont{Iansui 094}
\usepackage[a4paper, margin=1.5cm]{geometry}
\usepackage{graphicx}
\usepackage{amsmath, amsfonts, amssymb}

\newcommand{\testtitle}[1]{ \begin{center}{\large\bf #1}\end{center} }

\begin{document}

\testtitle{怪怪數學題目的詳解}

\section{暖身題}

\begin{enumerate}
	\item 不行,有一個 7-連通塊中間有洞。
	\item 切成五條一樣的長條狀。
	\item 遇到要特別小心首項。\\
		\(a_n = \begin{cases}
			5,      & n = 1\\
			2n + 2, & n \ge 2
		\end{cases}\)
		\item $a_n = 3 \cdot (\frac{4}{3}) ^ {n-1}, \frac{1}{a_n} = \frac{1}{3} \cdot (\frac{3}{4}) ^ {n-1},\\ \sum _ {i = 1} ^ {\infty} a_i = \frac{1}{3} \cdot \frac{1 - (\frac{3}{4}) ^ {\infty}}{1 - (\frac{3}{4})} = \frac{4}{3}$
\end{enumerate}

\section{通靈消消消}

\begin{enumerate}
	\item \begin{align*}
			&\sqrt[256]{(2+1)(2^2+1)(2^4+1)\cdots(2^{256}+1)+1}\\
			= &\sqrt[256]{(2-1)(2+1)(2^2+1)(2^4+1)\cdots(2^{256}+1)+1}\\
			= &\sqrt[256]{(2^2-1)(2^2+1)(2^4+1)\cdots(2^{256}+1)+1}\\
			= &\sqrt[256]{(2^4-1)(2^4+1)\cdots(2^{256}+1)+1}\\
			= &\sqrt[256]{(2^{512}-1)+1}\\
			= & 4
		\end{align*}

	\item \begin{align*}
			    S_n  & = & k & + & k ^ 2 & + & \cdots & + & k ^ n & \\
			    kS_n & = &   &   & k ^ 2 & + & \cdots & + & k ^ n & + k ^ {n+1}\\
			(1-k)S_n & = & k & - & k ^ {n + 1}\\
			\Rightarrow S_n & = \frac{k - k ^ {n+1}}{1-k}
		\end{align*}

	\item \begin{align*}
			    S_n  & = & k & + & 2k ^ 2 & + & \cdots + & nk ^ n & \\
					kS_n & = &   &   &  k ^ 2 & + & \cdots + & (n-1)k ^ n & + nk ^ {n+1}\\
			(1-k)S_n & = & k & + &  k ^ 2 & + & \cdots + & k ^ n      & - nk ^ {n+1} \\
		\end{align*}
	\begin{align*}
		\Rightarrow (1-k)S_n & = \frac{k - k ^ {n+1}}{1-k} - nk ^ {n+1}\\
		\Rightarrow S_n & = \frac{k - k ^ {n+1} - (1-k)nk ^ {n+1}}{(1-k)^2}
		\end{align*}

	\item 把整個圖左右翻轉和原本的加起來,這樣第 $k$ 行的每個數字都是 $2 ^ {k-1} (n + 2)$ ,第 $n$ 行的那個數字就是 $2 ^ n (n + 2)$ ,除以二即得答案 $2 ^ {n-1} (n + 2)$ 。
\end{enumerate}

\pagebreak

\section{進階的等比數列}

\begin{enumerate}
	\item $a_n = 2^n$
	\item 移項遞迴式得 $a_n - a_{n-1} = 2a_{n-1} - 2a_{n-2} = 2(a_{n-1} - a_{n-2})$ ,\\
		令 $b_n = a_{n+1} - a_n$ ,得 $b_1 = 2, b_n = 2b_{n-1}$ ,\\
		故 $b_n = 2^n, a_n = b_{n-1} + a_{n-1}\\ \Rightarrow a_n = a_1 + b_1 + b_2 + \cdots + b_{n-1} = 5 + 2 + 4 + \cdots + 2 ^ {n-1} = 4 + 2^n$
	\item 先把礙眼的根號除掉。\\
		令 $b_n = \sqrt{a_n}$ ,得 $b_1 = 1, b_2 = 2, b_n b_{n-2} = 2 b_{n-1} b_{n-2} + p b_{n-1} ^ 2$ 。\\
		兩邊同除 $b_{n-1} b_{n-2}$ ,得 $\frac{b_n}{b_{n-1}} = 2 + p \frac{b_{n-1}}{b_{n-2}}$ 。\\
		令 $c_n = \frac{b_{n+1}}{b_n}$ ,得 $c_1 = 2, c_n = pc_{n-1} + 2 \Rightarrow c_n = p(p(\dots(p \cdot c_1 + 2) \dots)+2)+2\\
		= 2 + 2p + 2p^2 + \dots + 2p^{n-1}
		= \begin{cases}
			2n, & p = 1\\
			2\frac{1-p^n}{1-p}, & p \ne 1
		\end{cases}$ ,又 $b_n = c_{n-1} \cdot b_{n-1} = b_1 \cdot c_1 \cdot c_2 \cdots c_{n-1}$ ,\\
		$\Rightarrow b_n = \begin{cases}
			2^{n-1} (n-1)!, & p = 1\\
			2^{n-1} \frac{\prod _ {k = 1} ^ {n - 1} (1 - p^k)}{(1-p)^{n-1}}, & p \ne 1
		\end{cases}$ ,平方即得 $a_n$ 。
\end{enumerate}

\section{奇怪的作法}

\begin{enumerate}
	\item 作法很神奇。令 $x = \sqrt{12 + x}$ ,解得 $x^2 = 12 + x, (x - 4)(x+3) = 0, x = 4, -3$ ,又 $x$ 必為正數,故 $x = 4$ 。
	\item 令 $\lim _ {n \rightarrow \infty} \frac{F_n}{F_{n-1}} = x$ ,得 $F_{n+2} = x^2F_n, F_{n+1} = xF_n \Rightarrow x^2F_n = xF_n + F_n, x^2 = x + 1, x = \frac{1 \pm \sqrt 5}{2}$ ,檢查後發現只有正的答案符合條件,故答為 $\frac{1 + \sqrt 5}{2}$ 。
\end{enumerate}

\section{勇氣很重要}

\begin{enumerate}
	\item 把 $\cos 60x^\circ$ 列出來,發現是 $1, \frac{1}{2}, \frac{-1}{2}, -1, \frac{-1}{2}, \frac{1}{2}$ 的循環,丟回去會發現全部消掉,剩下 $1$ 。
\end{enumerate}

\end{document}
