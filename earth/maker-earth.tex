\documentclass[12pt]{article}

\usepackage[CJKnumber]{xeCJK}
\setCJKmainfont{Iansui 094}
\usepackage[a4paper, margin=1.5cm]{geometry}
\usepackage{graphicx}
\usepackage{multicol}
\usepackage{amsmath,amsfonts,amssymb}

\newcommand{\testtitle}[1]{ \begin{center}{\large\bf #1}\end{center} }

\begin{document}

\testtitle{很像智力測驗的地科}

\section{前言}

地科真的很像智力測驗,不要慌慢慢推就好。遇到沒看過的東西也不要慌張,文章通常會給定義。

\section{天文}

\subsection{溫故知新}

\begin{enumerate}
	\item 春分點:春分時太陽所在位置,黃道座標系將春分點定義為 $(0^\circ, 0^\circ)$ ,春分點在赤道座標系則為 $(0h, 0^\circ)$ 。
	\item 赤緯:可以想像成把緯度直接投影在天球上。
	\item 赤經:用時、分、秒來表示角度。以春分點為 0h ,順時針每 $15^\circ$ 為 1h 標示天球經度。
\end{enumerate}

\begin{center} \includegraphics*{a.pdf} \end{center}

已知圖上方為地球北方,請在圖上畫出:

\begin{enumerate}
	\item 地球自轉方向
	\item 地球公轉方向
	\item 分別將 $1 \sim 4$ 標上春分、夏至、秋分、冬至(以北半球為準)。
	\item 哪個星座位於春分點?
	\item 赤緯最接近 14h 的是哪個星座?
\end{enumerate}

\subsection{似乎是考古題?}

\begin{center} \includegraphics*[width=15cm]{b.pdf} \end{center}

\begin{multicols}{2}

\begin{tabular}{c|c|c|c}
	日期 & 編號 & 火星赤經 & 赤經換算角度 \\
	3/21 & 0 & 15h:33m & 233.25 \\
	4/06 & 1 & 15h:34m & 233.50 \\
	4/21 & 2 & 15h:22m & 230.50 \\
	5/06 & 3 & 15h:00m & ? \\
	5/21 & 4 & 14h:38m & 219.50 \\
	6/06 & 5 & 14h:25m & 216.25 \\
	6/21 & 6 & 14h:26m & 216.50 \\
	7/06 & 7 & 14h:38m & 219.50 \\
	7/21 & 8 & 14h:59m & ?
\end{tabular}
\columnbreak

此題可以使用量角器、直尺及圓規。

編號 $0 \sim 8$ 為地球某年時的位置,日期如左表所示。在表中所示的日期時,從地球觀測火星,得火星的赤經。假設地球及火星的軌道都是正圓,請回答以下問題:

\begin{enumerate}
	\item 請完成「赤經換算角度」問號部份。
	\item 請畫出地球於 $0~8$ 的位置時火星所在的方位的射線。
	\item 請畫出火星軌道,並求出火星的軌道半徑(以天文單位表示)。
	\item 火星逆行現象大約會維持多久(單位:月)?
\end{enumerate}

\end{multicols}

備註:實際上題目似乎是以真實發生的火星逆行做基礎出題,也沒有提供赤經換算角度。

\pagebreak

\section{怎麼記聖嬰現象}

\subsection{正常年}

從「赤道東風帶」開始一步一步建構$\cdots\cdots$

\vspace{4cm}

\begin{center} \includegraphics*[width=15cm]{c.pdf} \end{center}

\vspace{4cm}

\subsection{聖嬰年}

從「赤道東風帶減弱」開始一步一步建構$\cdots\cdots$

\vspace{4cm}

\begin{center} \includegraphics*[width=15cm]{c.pdf} \end{center}

\vspace{4cm}

\end{document}
